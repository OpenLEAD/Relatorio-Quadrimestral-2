%---------------------------------------------------------------------
\subsubsection{Minuta de reuni�o (07-abr-2014)}

\begin{tabbing}
  Local \= xxx \kill
  Local \> : LEAD \\
  Data  \> : 07 de Abril de 2014 \\
  Hora  \> : 10:00
\end{tabbing}

%---------------------------------------------------------------------
\participantes{
  %\alana,
  %\jacoud,
  %\andre,
  \elael,
  \gabriel,
  \julia,
  \patrick,
  %\ramon,  %Ausente. Visita do David Lane.
  \renan.
}

\pauta{Acompanhamento das atividades.}

\begin{itemize}
  \item Abertura. A reuni�o do Projeto ROSA foi convocada por \ramon.

  \item Aprova��o da minuta da reuni�o anterior. \\

  \item Em aberto:
  \begin{itemize}
    \item Discuss�o t�cnica a respeito do sistema operacional a ser empregado e status dos Materiais Permanentes.
    \begin{itemize}
      \item Pedidos para compra de Sonar e Pan\&Tilt est�o em andamento.
      \item Encoder. Entrar em contato com a empresa que ainda n�o respondeu.
      \item Sensor Indutivo. Colocar uma mola para garantir complac�ncia. O fornecedor solicitou um pedido formal de compra com CNPJ, Raz�o Social e Endere�o de Entrega. Fornecedores: PEPPERL + FLUX (conferir se j� t�m cadastro).
      \item Eletr�nica embarcada (Patrick e Ramon).
    \end{itemize}
  \end{itemize}

  \item Prot�tipos:
  \begin{itemize}
    \item ROCK x Android: tempo e ambiente diferentes. ROCK n�o precisa ser online. Android precisaria de Wi-Fi. \\
    \item Qual tablet ser� usado? Encontrar um tablet que funciona pra os dois ou um tablet especifico. UBUNTU e Android se comunicam? Existe um tablet que ofere�a uma escalabilidade? Precisamos decidir. \\
    \item Elael: Decidir com Sylvain como faremos o aplicativo. \\
    \item O Android � mais `bonito' do que o QT e tem uma poss�vel modularidade. Por exemplo, pode-se executar aplicativos que rodariam em celular. O compromisso � que precisar�amos de uma conex�o online. \\
    %\item J� o QT e n�o sair do ambiente ROCK, componentes falando com componentes. (???) \\
    \item Vantagens do Android: 1) � r�pido de implementar e 2) tem QT para Android, embora n�o seja est�vel, tem muitos bugs. Tamb�m existem softwares para desenvolvimento de Android
        completamente integrados (interface, funcionalidade) que aparentemente ainda n�o existe para QT. \\
    \item O sistema operacional do Ubuntu pode inutilizar o tablet \\ (https://wiki.ubuntu.com/Touch/Install).  \\
  \end{itemize}

  \item Tomadas de decis�o:
  \begin{itemize}
    \item Eletr�nica Embarcada (GTR Company).
    \item Housing
    \item Sonar ESBR (Contrato?) \\
  \end{itemize}

  \item \textbf{\andre:}
  \begin{itemize}
    \item Trabalhando na parte matem�tica da bateria.
    \item Estudo de par�metro de Bateria e suas cargas.
    \item Filtros de Kalman tem limita��es de n�o-linearidade. \\
  \end{itemize}

  \item Defini��o dos semin�rios:
    \begin{itemize}
      \item 13/03 (4a.-feira), 10:00 �s 12:00.
      \item 18/03 (3a.-feira), 10:00 �s 12:00
      \item 20/03 (5a.-feira), 10:00 �s 12:00.
      \item 21/03 (6a.-feira), 13:00 �s 15:00.
    \end{itemize}

%  \item Pauta para a pr�xima reuni�o: N�o definida.

\end{itemize}

\vspace{10mm}%
\parbox[t]{70mm}{
  Aprovado por: \\[5mm]
  \centering
  \includegraphics[width=65mm]{../assinatura/assinatura-digital.jpg} \\[-4mm]
  \rule[2mm]{70mm}{0.1mm} \\
  \ramon \\[1mm]
  Coordenador do Projeto \\
}

%---------------------------------------------------------------------
\fim
