%---------------------------------------------------------------------
\subsubsection{Minuta de reuni�o (17-dez-2013)}

\begin{tabbing}
  Local \= xxx \kill
  Local \> : LEAD \\
  Data  \> : 17 de Dezembro de 2013 \\
  Hora  \> : 10:00
\end{tabbing}

%---------------------------------------------------------------------
\participantes{
  \jacoud,
  \andre,
  \elael,
  \gabriel,
  \julia,
  \ramon,
  \renan.
}

\pauta{Acompanhamento das atividades.}

\begin{itemize}
  \item Abertura. A reuni�o do Projeto ROSA foi convocada por \ramon.

  \item Aprova��o da minuta da reuni�o anterior.

  \item Em aberto:
  \begin{itemize}
    \item Assinar of�cios para obter as rubricas.
    \item Ajuste no seguro de vida.
    \item Encontrar substituto para o Rafael.
    \item Encontrar/contratar engenheiro mec�nico.
    \item Requerimentos para aquisi��o de laptops LENOVO, sensor, sonar e encoder, nacionais e importados.
    \item Entregar requerimentos de compras para sonar (ESBR), sensor e codificador
    \item Contrato de transfer�ncia para Alemanha do Renan finalizado.
  \end{itemize}

  \item Tarefas para o Jacoud:
  \begin{itemize}
    \item Coordenar quest�es e tarefas para \andre.
    \item Introduzir ScrumDo como refer�ncia para nosso projeto.
  \end{itemize}

  \item Sylvain:
  \begin{itemize}
    \item N�o participar� da reuni�o pois n�o temos internet.
  \end{itemize}

  \item Grupo de design:
  \begin{itemize}
  \item \textbf{\julia.} Coordenou entrega de Relat�rio Mensal, Relat�rio de Usabilidade e de toda documenta��o para a ESBR. Quest�es administrativas em andamento. Fez apresenta��o de prot�tipos e testes que podem ser usados no projeto.
  \end{itemize}

  \item Grupo de software:
  \begin{itemize}
    \item \textbf{\gabriel.} Est� trabalhando no que foi recomendado pelo Sylvain. Construiu um componente Orogen para fazer um tipo opaco do Octomap/Octree que se comunica com o Plug-in. Fez apresenta��o do Octoviz para explicar no que tem trabalhado.

    \item \textbf{\elael.} Fechou o driver em ROCK com a ressalva de que o pixel buffer ainda esta inst�vel. Corrigiu o problema do Z-Buffer. Pr�ximo passo � avan�ar no component ROCK.
  \end{itemize}

  \item Grupo de pot�ncia:
  \begin{itemize}
    \item \textbf{\andre.} Precisa de direcionamento maior para que possa continuar na pesquisa. Aguarda o feedback.  Alessandro vai coordenar algumas tarefas para direcionar essa pesquisa.

    \item \textbf{\renan.} Aguardando o feedback do Patrick a respeito do Sonar e da proforma do Sensor indutivo (importado). Marcou reuni�o com o pessoal da Tritec. Fez apresenta��o com um resumo do que j� pesquisou, � o que est� em aberto. Tamb�m fez um sketch sobre a eletr�nica geral do projeto (evolu��o de acordo com as mudan�as) assim como se dedicou ao power supply. Conversou com Igor (Projeto DORIS) sobre a possibilidade de usar uma bateria embarcada ao inv�s de um umbilical.
  \end{itemize}

%  \item Pauta para a pr�xima reuni�o: N�o definida.

\end{itemize}

\vspace{10mm}%
\parbox[t]{70mm}{
  Aprovado por: \\[5mm]
  \centering
  %\includegraphics[bb=1 1 1238 299,width=65mm]{../assinatura/assinatura-digital.jpg} \\[-4mm]
  \includegraphics[width=65mm]{../assinatura/assinatura-digital.jpg} \\[-4mm]
  \rule[2mm]{70mm}{0.1mm} \\
  \ramon \\[1mm]
  Coordenador do Projeto \\
}

%---------------------------------------------------------------------
\fim
